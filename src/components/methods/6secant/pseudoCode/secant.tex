\documentclass[12pt]{article}
\usepackage{amsmath}
\usepackage{latexsym}
\usepackage{amsfonts}
\usepackage[normalem]{ulem}
\usepackage{soul}
\usepackage{array}
\usepackage{amssymb}
\usepackage{extarrows}
\usepackage{graphicx}
\usepackage[backend=biber,
style=numeric,
sorting=none,
isbn=false,
doi=false,
url=false,
]{biblatex}\addbibresource{bibliography.bib}

\usepackage{subfig}
\usepackage{wrapfig}
\usepackage{wasysym}
\usepackage{enumitem}
\usepackage{adjustbox}
\usepackage{ragged2e}
\usepackage[svgnames,table]{xcolor}
\usepackage{tikz}
\usepackage{longtable}
\usepackage{changepage}
\usepackage{setspace}
\usepackage{hhline}
\usepackage{multicol}
\usepackage{tabto}
\usepackage{float}
\usepackage{multirow}
\usepackage{makecell}
\usepackage{fancyhdr}
\usepackage[toc,page]{appendix}
\usepackage[hidelinks]{hyperref}
\usetikzlibrary{shapes.symbols,shapes.geometric,shadows,arrows.meta}
\tikzset{>={Latex[width=1.5mm,length=2mm]}}
\usepackage{flowchart}\usepackage[paperheight=11.0in,paperwidth=8.5in,left=1.18in,right=1.18in,top=0.98in,bottom=0.98in,headheight=1in]{geometry}
\usepackage[utf8]{inputenc}
\usepackage[T1]{fontenc}
\TabPositions{0.5in,1.0in,1.5in,2.0in,2.5in,3.0in,3.5in,4.0in,4.5in,5.0in,5.5in,6.0in,}

\urlstyle{same}

\renewcommand{\_}{\kern-1.5pt\textunderscore\kern-1.5pt}

 %%%%%%%%%%%%  Set Depths for Sections  %%%%%%%%%%%%%%

% 1) Section
% 1.1) SubSection
% 1.1.1) SubSubSection
% 1.1.1.1) Paragraph
% 1.1.1.1.1) Subparagraph


\setcounter{tocdepth}{5}
\setcounter{secnumdepth}{5}


 %%%%%%%%%%%%  Set Depths for Nested Lists created by \begin{enumerate}  %%%%%%%%%%%%%%


\setlistdepth{9}
\renewlist{enumerate}{enumerate}{9}
		\setlist[enumerate,1]{label=\arabic*)}
		\setlist[enumerate,2]{label=\alph*)}
		\setlist[enumerate,3]{label=(\roman*)}
		\setlist[enumerate,4]{label=(\arabic*)}
		\setlist[enumerate,5]{label=(\Alph*)}
		\setlist[enumerate,6]{label=(\Roman*)}
		\setlist[enumerate,7]{label=\arabic*}
		\setlist[enumerate,8]{label=\alph*}
		\setlist[enumerate,9]{label=\roman*}

\renewlist{itemize}{itemize}{9}
		\setlist[itemize]{label=$\cdot$}
		\setlist[itemize,1]{label=\textbullet}
		\setlist[itemize,2]{label=$\circ$}
		\setlist[itemize,3]{label=$\ast$}
		\setlist[itemize,4]{label=$\dagger$}
		\setlist[itemize,5]{label=$\triangleright$}
		\setlist[itemize,6]{label=$\bigstar$}
		\setlist[itemize,7]{label=$\blacklozenge$}
		\setlist[itemize,8]{label=$\prime$}

\setlength{\topsep}{0pt}\setlength{\parskip}{8.04pt}
\setlength{\parindent}{0pt}

 %%%%%%%%%%%%  This sets linespacing (verticle gap between Lines) Default=1 %%%%%%%%%%%%%%


\renewcommand{\arraystretch}{1.3}


%%%%%%%%%%%%%%%%%%%% Document code starts here %%%%%%%%%%%%%%%%%%%%



\begin{document}
{\fontsize{16pt}{19.2pt}\selectfont \textbf{\uline{Secant }}\par}\par

$\ast$ The user should guarantee that the function f(X) is continuous. \par

\begin{enumerate}
	\item Ask the user for the function f(X), the tolerance, and the maximum number of iterations. \par

	\item Ask the user for two initial values we will call X\textsubscript{0 }and X\textsubscript{1 }\par

	\item We evaluate X\textsubscript{0 }and X\textsubscript{1 }in the function to obtain f(X\textsubscript{0}) and f(X\textsubscript{1}). If f(X\textsubscript{0}) = 0 or f(X\textsubscript{1}) = 0, then we will tell the user that this is the root\par

	\item Now we find the following value of X by storing it in the variable X\textsubscript{n }$ \ldots $  we find it using the following formula : X\textsubscript{n }= X\textsubscript{1 }– [f(X\textsubscript{1})$\ast$ ( X\textsubscript{1 }– X\textsubscript{0 }) / f(X\textsubscript{1}) – f(X\textsubscript{0})] then we evaluate this in the function f(X)$ \ldots $  if this answer gives zero, then we alert the user that this is the root. \par

	\item We make an iterations counter, calling it n and we start with 1\par

	\item Error = $ \vert $  X\textsubscript{1 }– X\textsubscript{n\  }$ \vert $ \textsubscript{  }\par

	\item Now we make a cycle$ \ldots $ . While the error > tolerance, f(X\textsubscript{n}) $ \neq $  0, n<max number of iterations, do: \par

\begin{enumerate}
	\item X\textsubscript{0 }= X\textsubscript{1 }\par

	\item X\textsubscript{1 }= X\textsubscript{n }\par

	\item f(X\textsubscript{0}) = the new value of X\textsubscript{0} evaluated in the function \par

	\item f(X\textsubscript{1}) = the new value of X\textsubscript{1 }evaluated in the function\par

	\item X\textsubscript{n }= X\textsubscript{1 }– [f(X\textsubscript{1})$\ast$ ( X\textsubscript{1 }– X\textsubscript{0 }) / f(X\textsubscript{1}) – f(X\textsubscript{0})] (with the new values)\par

	\item f(X\textsubscript{n}) = the new value of X\textsubscript{n }evaluated in the function \par

	\item n = n+1 \par

\begin{enumerate}
	\item inside the function we put a conditional , that if f(X\textsubscript{1}) – f(X\textsubscript{0}) = 0, as the denominator will be zero we tell the user that he cannot continue with the method given that the denominator is zero and that the root is approximately X\textsubscript{n }(final value) with an error of :\_\_\_ (with the final value of the error)\par


\end{enumerate}
\end{enumerate}
	\item If the error $ \leq $  tolerance, tell the user that the root is approximately X\textsubscript{n }(final\ value) with an error of \_\_\_\_ (with the final value of the error)\par

	\item If f(X\textsubscript{n}) = 0 tell the user that X\textsubscript{n }is the root. \par

	\item If n is equal to the number of iterations tell the user that he has reached the maximum number of iterations and that the root will be approximately X\textsubscript{n }(final value) with an error of\_\_\_\_(with the final value of the error)
\end{enumerate}\par


\vspace{\baselineskip}

\printbibliography
\end{document}