\documentclass[12pt]{article}
\usepackage{amsmath}
\usepackage{latexsym}
\usepackage{amsfonts}
\usepackage[normalem]{ulem}
\usepackage{soul}
\usepackage{array}
\usepackage{amssymb}
\usepackage{extarrows}
\usepackage{graphicx}
\usepackage[backend=biber,
style=numeric,
sorting=none,
isbn=false,
doi=false,
url=false,
]{biblatex}\addbibresource{bibliography.bib}

\usepackage{subfig}
\usepackage{wrapfig}
\usepackage{wasysym}
\usepackage{enumitem}
\usepackage{adjustbox}
\usepackage{ragged2e}
\usepackage[svgnames,table]{xcolor}
\usepackage{tikz}
\usepackage{longtable}
\usepackage{changepage}
\usepackage{setspace}
\usepackage{hhline}
\usepackage{multicol}
\usepackage{tabto}
\usepackage{float}
\usepackage{multirow}
\usepackage{makecell}
\usepackage{fancyhdr}
\usepackage[toc,page]{appendix}
\usepackage[hidelinks]{hyperref}
\usetikzlibrary{shapes.symbols,shapes.geometric,shadows,arrows.meta}
\tikzset{>={Latex[width=1.5mm,length=2mm]}}
\usepackage{flowchart}\usepackage[paperheight=11.0in,paperwidth=8.5in,left=1.18in,right=1.18in,top=0.98in,bottom=0.98in,headheight=1in]{geometry}
\usepackage[utf8]{inputenc}
\usepackage[T1]{fontenc}
\TabPositions{0.5in,1.0in,1.5in,2.0in,2.5in,3.0in,3.5in,4.0in,4.5in,5.0in,5.5in,6.0in,}

\urlstyle{same}

\renewcommand{\_}{\kern-1.5pt\textunderscore\kern-1.5pt}

 %%%%%%%%%%%%  Set Depths for Sections  %%%%%%%%%%%%%%

% 1) Section
% 1.1) SubSection
% 1.1.1) SubSubSection
% 1.1.1.1) Paragraph
% 1.1.1.1.1) Subparagraphhttps://www.overleaf.com/project/5eb9bfa3b7d1ef00014d5a4a


\setcounter{tocdepth}{5}
\setcounter{secnumdepth}{5}


 %%%%%%%%%%%%  Set Depths for Nested Lists created by \begin{enumerate}  %%%%%%%%%%%%%%


\setlistdepth{9}
\renewlist{enumerate}{enumerate}{9}
		\setlist[enumerate,1]{label=\arabic*)}
		\setlist[enumerate,2]{label=\alph*)}
		\setlist[enumerate,3]{label=(\roman*)}
		\setlist[enumerate,4]{label=(\arabic*)}
		\setlist[enumerate,5]{label=(\Alph*)}
		\setlist[enumerate,6]{label=(\Roman*)}
		\setlist[enumerate,7]{label=\arabic*}
		\setlist[enumerate,8]{label=\alph*}
		\setlist[enumerate,9]{label=\roman*}

\renewlist{itemize}{itemize}{9}
		\setlist[itemize]{label=$\cdot$}
		\setlist[itemize,1]{label=\textbullet}
		\setlist[itemize,2]{label=$\circ$}
		\setlist[itemize,3]{label=$\ast$}
		\setlist[itemize,4]{label=$\dagger$}
		\setlist[itemize,5]{label=$\triangleright$}
		\setlist[itemize,6]{label=$\bigstar$}
		\setlist[itemize,7]{label=$\blacklozenge$}
		\setlist[itemize,8]{label=$\prime$}

\setlength{\topsep}{0pt}\setlength{\parskip}{8.04pt}
\setlength{\parindent}{0pt}

 %%%%%%%%%%%%  This sets linespacing (verticle gap between Lines) Default=1 %%%%%%%%%%%%%%


\renewcommand{\arraystretch}{1.3}


%%%%%%%%%%%%%%%%%%%% Document code starts here %%%%%%%%%%%%%%%%%%%%



\begin{document}
\textbf{\uline{Newton Divided Differences }}\par


\vspace{\baselineskip}
\begin{enumerate}
	\item Given a table of (x,y) pairs, input x values as: x\textsubscript{0},x\textsubscript{1},x\textsubscript{2},x\textsubscript{3}$ \ldots $ .x\textsubscript{n}, and input the y values and assign them to an f(x):  f(x\textsubscript{0}),f(x\textsubscript{1}),f(x\textsubscript{2})$ \ldots $ .f(x\textsubscript{n}) as F\textsubscript{0,0, }F\textsubscript{1},\textsubscript{0},$ \ldots $ .F\textsubscript{n,0}\par

	\item Output the numbers F\textsubscript{0,0}, F\textsubscript{1,1 }$ \ldots $ . F\textsubscript{n,n }where the summation: \par

\begin{enumerate}
	\item P(x)=  \(  \sum _{x=0}^{n}F\textsubscript{i,i} \prod_{j=0}^{i-1} \left( x-x\textsubscript{j} \right)  \) \par


\end{enumerate}
	\item Time for cycle:\par

\begin{enumerate}
	\item For i = 1 < n, i++\par

\begin{enumerate}
	\item For j = 1 < i, j++ \par

\begin{enumerate}
	\item Set F\textsubscript{ij} = (F\textsubscript{ij-1}- F \textsubscript{i-1,j-1})/x\textsubscript{i}-x\textsubscript{i-j}\par


\end{enumerate}
\end{enumerate}
	\item Output (F\textsubscript{00},F\textsubscript{11}$ \ldots $ ..,F\textsubscript{nn}); (F\textsubscript{ii }is f[x\textsubscript{0},x\textsubscript{1}$ \ldots $ ..x\textsubscript{i}])\par


\end{enumerate}
	\item Stop 
\end{enumerate}\par


\printbibliography
\end{document}