\documentclass[12pt]{article}
\usepackage{amsmath}
\usepackage{latexsym}
\usepackage{amsfonts}
\usepackage[normalem]{ulem}
\usepackage{soul}
\usepackage{array}
\usepackage{amssymb}
\usepackage{extarrows}
\usepackage{graphicx}
\usepackage[backend=biber,
style=numeric,
sorting=none,
isbn=false,
doi=false,
url=false,
]{biblatex}\addbibresource{bibliography.bib}

\usepackage{subfig}
\usepackage{wrapfig}
\usepackage{wasysym}
\usepackage{enumitem}
\usepackage{adjustbox}
\usepackage{ragged2e}
\usepackage[svgnames,table]{xcolor}
\usepackage{tikz}
\usepackage{longtable}
\usepackage{changepage}
\usepackage{setspace}
\usepackage{hhline}
\usepackage{multicol}
\usepackage{tabto}
\usepackage{float}
\usepackage{multirow}
\usepackage{makecell}
\usepackage{fancyhdr}
\usepackage[toc,page]{appendix}
\usepackage[hidelinks]{hyperref}
\usetikzlibrary{shapes.symbols,shapes.geometric,shadows,arrows.meta}
\tikzset{>={Latex[width=1.5mm,length=2mm]}}
\usepackage{flowchart}\usepackage[paperheight=11.0in,paperwidth=8.5in,left=1.18in,right=1.18in,top=0.98in,bottom=0.98in,headheight=1in]{geometry}
\usepackage[utf8]{inputenc}
\usepackage[T1]{fontenc}
\TabPositions{0.5in,1.0in,1.5in,2.0in,2.5in,3.0in,3.5in,4.0in,4.5in,5.0in,5.5in,6.0in,}

\urlstyle{same}

\renewcommand{\_}{\kern-1.5pt\textunderscore\kern-1.5pt}

 %%%%%%%%%%%%  Set Depths for Sections  %%%%%%%%%%%%%%

% 1) Section
% 1.1) SubSection
% 1.1.1) SubSubSection
% 1.1.1.1) Paragraph
% 1.1.1.1.1) Subparagraph


\setcounter{tocdepth}{5}
\setcounter{secnumdepth}{5}


 %%%%%%%%%%%%  Set Depths for Nested Lists created by \begin{enumerate}  %%%%%%%%%%%%%%


\setlistdepth{9}
\renewlist{enumerate}{enumerate}{9}
		\setlist[enumerate,1]{label=\arabic*)}
		\setlist[enumerate,2]{label=\alph*)}
		\setlist[enumerate,3]{label=(\roman*)}
		\setlist[enumerate,4]{label=(\arabic*)}
		\setlist[enumerate,5]{label=(\Alph*)}
		\setlist[enumerate,6]{label=(\Roman*)}
		\setlist[enumerate,7]{label=\arabic*}
		\setlist[enumerate,8]{label=\alph*}
		\setlist[enumerate,9]{label=\roman*}

\renewlist{itemize}{itemize}{9}
		\setlist[itemize]{label=$\cdot$}
		\setlist[itemize,1]{label=\textbullet}
		\setlist[itemize,2]{label=$\circ$}
		\setlist[itemize,3]{label=$\ast$}
		\setlist[itemize,4]{label=$\dagger$}
		\setlist[itemize,5]{label=$\triangleright$}
		\setlist[itemize,6]{label=$\bigstar$}
		\setlist[itemize,7]{label=$\blacklozenge$}
		\setlist[itemize,8]{label=$\prime$}

\setlength{\topsep}{0pt}\setlength{\parskip}{8.04pt}
\setlength{\parindent}{0pt}

 %%%%%%%%%%%%  This sets linespacing (verticle gap between Lines) Default=1 %%%%%%%%%%%%%%


\renewcommand{\arraystretch}{1.3}


%%%%%%%%%%%%%%%%%%%% Document code starts here %%%%%%%%%%%%%%%%%%%%



\begin{document}
\textbf{\uline{Crout Factorization }}\par

\begin{enumerate}
	\item Ask the user for a matrix A. Needs to be a squared matrix. \par

	\item Ask the use for vector b and make sure that is the same length as matrix A, and ask for a whole number n. \par

	\item Next, we create 2 more matrices, we call them L and U. Matrices L and U need to have the following characteristics\par

\begin{enumerate}
	\item Have the same size as matrix A \par

	\item L will have the same elements that are below the diagonal in A but with the opposite sign. The diagonal in L will be the same as A, and all other elements will be 0. \par

	\item U will have the same elements in A that are above the diagonal with opposite signs, and will have the same elements in the diagonal as A and the rest of the elements will be 0.\par


\vspace{\baselineskip}

\end{enumerate}
	\item Time for a cycle\par

\begin{enumerate}
	\item For k = 1 < n, with 1 step (k++)\par

\begin{enumerate}
	\item sum1 = 0\par

	\item For p = 1 < k-1, with 1 step (p++)\par

	\item sum1 = sum1 + L\textsubscript{kp}$\ast$  U\textsubscript{kp}\par

	\item L\textsubscript{kk }= \( \sqrt[]{a\textsubscript{kk}+sum1} \) \par

	\item For i = k+1 < n , with 1 step (i++) \par

	\item sum2 = 0\par

\begin{enumerate}
	\item For r = 1 < k-1 , with 1 step (r++) \par

	\item sum2=sum2+L\textsubscript{ir }$\ast$  U\textsubscript{rk }\par


\end{enumerate}
	\item L\textsubscript{ik} = (a\textsubscript{kk} $-$  sum2)\par


\end{enumerate}
	\item For j = k+1 < n , with 1 step (j++) \par

\begin{enumerate}
	\item sum3 = 0\par

\begin{enumerate}
	\item For s = 1 < s-1, with 1 step (s++) \par

\begin{enumerate}
	\item sum3 = sum3 + L\textsubscript{ks}$\ast$ U\textsubscript{sj}\par

	\item U\textsubscript{kj} =(a\textsubscript{kj}$-$ sum3)/ L\textsubscript{kk}\par


\end{enumerate}
\end{enumerate}
\end{enumerate}
\end{enumerate}
	\item z = progressive substitution (1,b) x = regressive substitution (u,z)\par

	\item Regressive Replacement \par

\begin{enumerate}
	\item Para i = n < 1 , with 1 step (i++) \par

\begin{enumerate}
	\item sum = 0\par

\begin{enumerate}
	\item for j = i +1 < n , with 1 step (j++) \par

\begin{enumerate}
	\item sum = sum+a\textsubscript{ij}$\ast$ x\textsubscript{ij}\par


\end{enumerate}
	\item xi =(a\textsubscript{in+1}$-$ sum)/a\textsubscript{ii}\par


\end{enumerate}
\end{enumerate}
\end{enumerate}
	\item Progressive replacement \par

\begin{enumerate}
	\item Para i = 1 < n , with 1 step (i++) \par

\begin{enumerate}
	\item sum = 0\par

\begin{enumerate}
	\item for j = i +1 < n , with 1 step (j++) \par

\begin{enumerate}
	\item sum = sum+a\textsubscript{ij}$\ast$ x\textsubscript{ij}\par


\end{enumerate}
	\item xi =(a\textsubscript{in+1}$-$ sum)/a\textsubscript{ii}\par


\vspace{\baselineskip}

\end{enumerate}
\end{enumerate}
\end{enumerate}
\end{enumerate}
\printbibliography
\end{document}